\section*{Sonntagslieder im 1. Modus}
\subsection*{Zur großen Vesper}\label{subsec:zu-vesper}

\textcolor{red}{\textit{Verslieder der Auferstehung}}

Unser Abendgebet, nimm an, Heiliger Herr \\
und vergib uns unsere Sünden, \\
denn du allein hast der Welt die Auferstehung erzeigt.

Umringt, ihr Völker, den Zion \\
und schließt\footnote{Alternativ: umschließt} ihn ringsumher \\
und gebt Ehre in ihm, \\
DEM, DER vom Tode auferstand!\\
Denn Er ist unser Gott, \\
DER uns befreit hat von all unserer Schuld.

Kommt, ihr Völker,
lasst uns singen und niederfallen vor Christo, \\
da wir preisen seine Auferstehung von den Toten! \\
Denn Er ist unser Gott, \\
der die Welt erlöst hat vom Trug(e) des Feindes.

\textcolor{red}{\textit{Verslieder von Anatolien}}

Jubelt, ihr Himmel! \\
Tönet, ihr Tiefen der Erde! \\
Ihr Berge, frohlocket mit Jauchzen! \\
Denn siehe, der Emmanuel hat unsere Sünden ans Kreuz genagelt, \\
und DER da Leben gibt, hat den Tod vernichtet \\
und Adam auferweckt, in Seiner Menschenliebe. \\

Der im Fleisch(e) sich freiwillig für uns kreuzigen ließ, \\
der gelitten hat und begraben wurde \\
und von den Toten auferstand, \\
Ihn loben wir und rufen also: \\
Stärke durch den wahren Glauben, deine Kirche, o Christ(e) \\
und befriede unser Leben, \\
denn du bist gütig und menschenliebend.
