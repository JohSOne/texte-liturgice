\section*{Sonntagslieder im 1. Modus}
\subsection*{Zur Vesper}\label{subsec:zu-vesper}

\textcolor{red}{\textit{Verslieder der Auferstehung}}

Unser Abendflehen, nimm an, Heiliger Herr und vergib uns unsere Sünden, denn du allein hast der Welt die Auferstehung offenbart.

Umringt, ihr Völker, den Zion und schließt ihn ringsumher, gebt Ehre in ihm, Dem, Der vom Tode auferstand!
Denn Er ist unser Gott, der uns befreit hat von unserer Schuld.

Kommt, ihr Völker, lasst uns Christum loben und anbeten, lasst uns\footnote{indem/da wir preise} preisen seine Auferstehung von den Toten!
Dies ist unser Gott, denn Er hat die Welt vom Trug des Feindes befreit.

\textcolor{red}{\textit{Verslieder von Anatolien}}

Jauchzet, ihr Himmel!
Jubelt, ihr Festen der Erde!
Ihr Berge, frohlocket mit Jauchzen!
Siehe\footnote{Denn siehe}, Emmanuel hat unsere Sünden ans Kreuz genagelt.
Der das Leben gibt, hat den Tod vernichtet.
Er hat Adam auferweckt, denn Er ist menschenliebend.
